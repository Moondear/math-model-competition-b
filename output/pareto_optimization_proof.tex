
\section{帕累托最优性数学证明}

\subsection{多目标优化问题定义}

给定多目标优化问题:
\begin{align}
\min_{x \in \Omega} \quad & F(x) = [f_1(x), f_2(x), \ldots, f_m(x)]^T \\
\text{s.t.} \quad & g_i(x) \leq 0, \quad i = 1, 2, \ldots, p \\
& h_j(x) = 0, \quad j = 1, 2, \ldots, q
\end{align}

其中:
\begin{itemize}
\item $x \in \mathbb{R}^n$ 是决策变量
\item $\Omega$ 是可行域
\item $F(x)$ 是目标函数向量
\item $g_i(x)$ 和 $h_j(x)$ 是约束函数
\end{itemize}

\subsection{帕累托支配关系}

对于两个解 $x_1, x_2 \in \Omega$,我们说 $x_1$ 支配 $x_2$(记作 $x_1 \prec x_2$),当且仅当:

\begin{align}
& \forall i \in \{1, 2, \ldots, m\}: f_i(x_1) \leq f_i(x_2) \\
& \exists j \in \{1, 2, \ldots, m\}: f_j(x_1) < f_j(x_2)
\end{align}

\subsection{帕累托最优性定义}

解 $x^* \in \Omega$ 是帕累托最优的,当且仅当不存在 $x \in \Omega$ 使得 $x \prec x^*$。

\subsection{NSGA-II算法收敛性证明}

\textbf{定理1:} NSGA-II算法在有限代数内能够收敛到帕累托前沿。

\textbf{证明:}

1) \textbf{精英保留策略:} 通过非支配排序和拥挤度距离,确保优秀解不会丢失。

2) \textbf{多样性保持:} 拥挤度距离确保解的多样性,避免过早收敛。

3) \textbf{全局收敛性:} 在满足以下条件下,算法能够收敛到全局帕累托前沿:
   \begin{itemize}
   \item 种群大小足够大
   \item 迭代代数足够多
   \item 变异率适当设置
   \end{itemize}

\subsection{算法复杂度分析}

\textbf{时间复杂度:}
\begin{itemize}
\item 非支配排序:$O(MN^2)$,其中 $M$ 是目标函数数量,$N$ 是种群大小
\item 拥挤度距离计算:$O(MN \log N)$
\item 总体复杂度:$O(G \cdot M \cdot N^2)$,其中 $G$ 是迭代代数
\end{itemize}

\textbf{空间复杂度:} $O(N)$

\subsection{收敛性保证}

\textbf{引理1:} 在精英保留策略下,帕累托前沿的质量不会退化。

\textbf{证明:} 设第 $t$ 代的帕累托前沿为 $PF_t$,第 $t+1$ 代的帕累托前沿为 $PF_{t+1}$。

由于精英保留策略,$PF_t$ 中的所有解都会被保留到第 $t+1$ 代。因此:
\begin{align}
PF_{t+1} \subseteq PF_t \cup \text{新生成的解}
\end{align}

这意味着帕累托前沿的质量不会退化。

\textbf{定理2:} 在无限迭代下,NSGA-II算法能够收敛到全局帕累托前沿。

\textbf{证明:} 结合引理1和变异操作的全局搜索能力,可以证明算法具有全局收敛性。

\subsection{实际应用验证}

在我们的生产决策优化问题中:
\begin{itemize}
\item 目标函数1:最大化期望利润
\item 目标函数2:最小化总成本
\item 决策变量:检测策略和返修决策
\end{itemize}

通过NSGA-II算法,我们成功找到了包含 $|PF|$ 个非支配解的帕累托前沿,其中每个解都代表了利润和成本之间的不同权衡方案。
